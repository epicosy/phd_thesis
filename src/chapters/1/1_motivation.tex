\section{Research Directions} \label{sec:se1}

In this thesis, we explore three main research directions as follows:

\subsection*{Automated Analysis and Mining of Vulnerabilities}

Contemporary vulnerability research remains siloed: fragmented taxonomies address only isolated aspects of faults (e.g., causes or resource constraints) without unifying code-level patterns and system-wide contexts, historical vulnerability repositories (e.g., NVD, OSV, GitHub Advisory) offer abundant data but lack automated tools for extracting and labeling actionable code vulnerabilities, and existing benchmarks (e.g., Defects4J, Vul4J) reproduce faults yet fail to mirror actual distributions across languages, technologies, and application domains. Our research direction seeks to bridge these gaps by developing an automated, data-driven framework that integrates multi‐level profiling of vulnerabilities, leverages scalable mining of historical feeds to identify and annotate code‐level weakness patterns, and synthesizes benchmarks whose composition reflects the true landscape of software faults. By aligning classification, automation, and benchmark construction, this approach aims to create a coherent ecosystem for understanding, evaluating, and ultimately improving software security.

\subsection*{Machine learning-based SAST}

Machine learning-based Static Application Security Testing (SAST) represents a paradigm shift from traditional rule-based vulnerability detection to data-driven approaches that can automatically learn patterns and characteristics of security vulnerabilities from large codebases. Unlike conventional SAST tools that rely on predefined rules and signatures to identify known vulnerability patterns, ML-based Software Vulnerability Prediction (SVP) systems leverage deep learning models, neural networks, and ensemble techniques to analyze source code and predict potential security weaknesses. These approaches promise to address fundamental limitations of traditional static analysis, including high false positive rates, limited adaptability to new vulnerability types, and the need for extensive expert knowledge to configure and maintain detection rules.

The motivation for exploring complementary trade-offs between SAST and Software Vulnerability Detection (SVD) approaches stems from their inherent strengths and limitations. Traditional SAST tools offer precision and explainability but struggle with novel vulnerability patterns, while ML-based SVD systems excel at pattern recognition but often lack interpretability and suffer from dataset biases. By investigating how these approaches can complement each other, this research aims to develop hybrid systems that combine rule-based precision with ML-based adaptability. Such integration could significantly reduce false positives while improving detection capabilities for emerging vulnerability types, ultimately creating more robust and practical security tools for real-world software development environments.

\subsection*{Benchmarking SAST-Based APR on Software Vulnerabilities}

The Open Source Software (OSS) ecosystem is a vibrant and collaborative environment comprising independent developers, enterprises, and academic contributors who work together to build software used in nearly every domain. Despite advances in development practices, security vulnerabilities remain a significant challenge, often discovered only after exploitation. Automated Program Repair (APR) for software vulnerabilities represents a promising approach to address this challenge, offering the potential to automatically generate patches for identified security weaknesses.

This research direction is motivated by the need to systematically evaluate and improve APR tools specifically for vulnerability remediation. By focusing on memory issues, data validation failures, and faulty computations in C, C++, and Java programs, this work aims to establish comprehensive benchmarks that reflect real-world vulnerability scenarios. The integration of Software Vulnerability Detection (SVD) techniques with APR systems has the potential to create more effective end-to-end solutions that not only identify vulnerabilities but also automatically generate appropriate fixes. Such integration could significantly reduce the security burden on developers and maintainers, enabling more proactive and systematic approaches to software security across the OSS ecosystem.

In this thesis, we also aim to answer the following high-level research questions:
\newline
\noindent
\textbf{\footnotesize RQ.1} \textit{How can automated vulnerability analysis shape models of software security and real-world benchmarks?}

The software security landscape is vast and complex, demanding a unified framework to profile faults across different layers—from languages to system components—to make sense of their diversity. However, profiling alone is not sufficient: to translate insights into practical benchmarks, we must isolate actionable vulnerabilities with concrete code artifacts that can be reproduced and tested. Achieving this at scale requires automated methods to mine historical vulnerability data, analyze source repositories, and label code patterns that signify real weaknesses. By combining comprehensive profiling with automated identification and annotation, we can construct benchmarks whose composition mirrors true fault distributions and supports reliable tool evaluation and comparative studies.
\newline
\newline
\textbf{\footnotesize RQ.2} \textit{What are the limitations of APR in repairing real-world software vulnerabilities, and can integrating SVD techniques like SAST help overcome them, particularly in coverage and fault localization?}

Despite significant advances in both Automated Program Repair (APR) and Software Vulnerability Detection (SVD) techniques such as Static Application Security Testing (SAST), a clear gap remains in understanding and improving their effectiveness on real-world software vulnerabilities. Existing studies often evaluate APR on general defects rather than security-specific issues, while SAST evaluations are frequently limited to synthetic benchmarks like SARD, raising concerns about generalizability. Moreover, APR suffers from persistent challenges like incomplete fault localization and overfitting to test cases, which undermine its reliability and scalability. SVD techniques, by contrast, can identify specific vulnerability patterns and provide semantic insights without execution, making them promising candidates to address APR’s bottlenecks. However, few empirical studies explore the integration of these techniques to enhance vulnerability coverage, localization accuracy, and patch quality. This research is motivated by the need to benchmark APR tools specifically on real-world vulnerabilities and to investigate whether incorporating SVD methods can mitigate known limitations, thereby advancing the state of vulnerability repair in practice.
