\section{Motivation \& Context} \label{sec:se1}

The Open Source Software (OSS) ecosystem is a vibrant and collaborative environment comprising independent developers, enterprises, and academic contributors who work together to build software used in nearly every domain, from operating systems to web applications. Tools such as version control systems, automated testing, and agile workflows have helped standardize and accelerate software development within this ecosystem. However, despite this progress, security is still often treated as an afterthought. Many projects remain vulnerable to security flaws until incidents prompt reactive fixes.

This thesis was motivated by the need to improve security in OSS projects proactively and systematically. The approach developed in this work enables early and automated detection and patching of vulnerabilities, helping OSS maintainers improve their code without needing specialized security expertise.

To that end, this work integrates existing static analysis and machine learning–based techniques into a unified automated framework. Using the strengths of Software Vulnerability Detection (SVD) and Automated Program Repair (APR) tools, we designed a system capable of identifying vulnerabilities and suggesting patches directly at the source code level. The platform developed in this research has the potential to be integrated into continuous integration pipelines and version control platforms like GitHub, where it could provide vulnerability alerts and patch suggestions in real-time, for every commit.
