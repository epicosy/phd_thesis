\subsection{Holistic Representation of Security Faults} \textbf{Aim:} Synthesize existing vulnerability taxonomies into a unified attribute schema. This phase consolidates diverse security-fault taxonomies (e.g. language-specific faults, technology categories, architectural layers, and root-cause taxonomies) to create a comprehensive representation of vulnerability attributes.
\newline

%Landwehr’s multi-dimensional foundation: NASA retains genesis (coding errors) and time (phase analysis) but replaces location with subsystem concentration (90% in 2–4 subsystems).

%Piessens’ causal focus: Extends this by linking vulnerability causes to lifecycle phases (e.g., Exception Management flaws stemming from ambiguous error handling during implementation).

%Weber’s tool-oriented schema: Complements by providing empirical data to train static analyzers on NASA’s dominant CWEs (e.g., Risky Values).


\textbf{Steps:}

\begin{itemize}
    \item \textbf{Survey taxonomies.} We collect relevant taxonomies from literature and standards. For example, language- and platform-based taxonomies (e.g. C/C++ errors vs. Web app issues), technology-focused taxonomies (e.g. web, database, OS level), software-layer taxonomies (e.g. presentation vs. business logic), and root-cause taxonomies (e.g. input validation, memory corruption).
    \item \textbf{Identify attributes.} From each taxonomy we extract key attributes (e.g., "Buffer Overflow", "SQL Injection", "Cross-Site Scripting", "Race Condition", etc.) and note which taxonomies include each attribute.
    \item \textbf{Create Unified Attribute Matrix.} We compile these attributes into a matrix with rows as example attributes and columns indicating taxonomy source. Table~\ref{tab:attribute-matrix} (excerpt below) illustrates how attributes map across taxonomies. This "Unified Attribute Matrix" highlights overlaps and gaps, guiding later analysis.
\end{itemize}

\begin{table}[ht!]
    \centering
    \caption{Unified Attribute Matrix}
    \label{tab:attribute-matrix}
    \begin{tabular}{lcccc}
        \hline
        \textbf{Attribute} & \textbf{Lang-based \cite{cwe_list}} & \textbf{Tech-based \cite{Khwaja2020ASF}} & \textbf{Layer \cite{sa_nikolai}} & \textbf{Root Cause \cite{Shahriar2012MitigatingPS}} \\
        \hline
        Buffer Overflow & \checkmark & \checkmark & & \checkmark \\
        SQL Injection & & \checkmark & \checkmark & \\
        Cross-site Scripting & & \checkmark & \checkmark & \\
        Privilege Escalation & \checkmark & & & \checkmark \\
        Invalid Input & & & & \checkmark \\
        \hline
    \end{tabular}
\end{table}

This phase yields a consolidated schema of security fault attributes that will guide data collection and later classification. In particular, the unified list of attributes becomes the basis for labeling and grouping vulnerabilities in subsequent phases.
