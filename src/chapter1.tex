\chapter{Introduction} \label{chap:intro}

\section{Motivation \& Context} \label{sec:se1}

The Open Source Software (OSS) ecosystem is a vibrant and collaborative environment comprising independent developers, enterprises, and academic contributors who work together to build software used in nearly every domain, from operating systems to web applications. Tools such as version control systems, automated testing, and agile workflows have helped standardize and accelerate software development within this ecosystem. However, despite this progress, security is still often treated as an afterthought. Many projects remain vulnerable to security flaws until incidents prompt reactive fixes.

This thesis was motivated by the need to improve security in OSS projects proactively and systematically. The approach developed in this work enables early and automated detection and patching of vulnerabilities, helping OSS maintainers improve their code without needing specialized security expertise.

To that end, this work integrates existing static analysis and machine learning–based techniques into a unified automated framework. Using the strengths of Software Vulnerability Detection (SVD) and Automated Program Repair (APR) tools, we designed a system capable of identifying vulnerabilities and suggesting patches directly at the source code level. The platform developed in this research has the potential to be integrated into continuous integration pipelines and version control platforms like GitHub, where it could provide vulnerability alerts and patch suggestions in real-time, for every commit.

\section{Research Gap} \label{sec:ae2}

While recent advances in SVD and Automated Program Repair APR have shown promise individually, significant challenges remain in their integration and systematic evaluation for security-critical applications. Current limitations span three interconnected areas:

\begin{enumerate}
    \item \textbf{Limited systematic integration between SVD and APR}: Although preliminary studies have explored using static analysis patterns for patch generation~\cite{Liu2018AVATARF,AlBataineh2021TowardsMR}, no comprehensive framework systematically leverages SVD outputs to guide APR fault localization and patch synthesis. Existing integration attempts focus on specific bug types rather than providing a unified approach for diverse vulnerability classes.
    \item \textbf{Evaluation gaps in security-specific contexts}: Current empirical evaluations suffer from two key limitations: (a) SVD tools are predominantly evaluated on synthetic datasets with known biases~\cite{Guo2024DataQI}, limiting generalizability to real-world vulnerability patterns; and (b) APR tools are primarily validated on generic functional bugs rather than security vulnerabilities, which exhibit distinct characteristics requiring specialized repair patterns~\cite{apr4vul}.
    \item \textbf{Limited understanding of trade-offs between detection techniques}: Traditional SAST tools and ML-based SVD approaches each have their strengths, granularity, and explainability versus speed and adaptability. However, few comparative or hybrid studies evaluate their combined effectiveness in detecting or patching vulnerabilities.
\end{enumerate}

This thesis addresses these gaps by developing the first systematic framework that integrates SVD and APR capabilities, validated through comprehensive empirical evaluation on real-world vulnerabilities, with explicit analysis of the detection-repair trade-offs.

\section{Problem Statement} \label{sec:ae3}

This thesis addresses the challenge of improving the repair capabilities of APR tools in the context of software vulnerabilities. While APR techniques have advanced significantly in recent years, they still struggle with certain classes of faults, particularly those related to security. These include vulnerabilities caused by memory mismanagement, incorrect data checks, and type-related computation errors in languages such as C, C++, and Java.

\eduard{Our investigation found that fault localization remains a critical bottleneck. For example... Additionally, the standard test-based validation mechanisms employed by these tools are often insufficient to confirm patch correctness in the context of vulnerabilities. APR tools still generated correct patches for only a limited subset of security faults across multiple open-source C projects. In our experiments, validation and fault localization were consistently among the top reasons for failed repairs.}

The problem, therefore, lies in the limitations of existing APR techniques to correctly and consistently patch vulnerabilities. This research explores whether integrating vulnerability detection into the repair workflow can enhance localization, reduce incorrect patching, and ultimately increase repair effectiveness.

\section{Aim \& Hypohtesis} \label{sec:ae4}

TODO

\section{Research Questions} \label{sec:ae4}

TODO

\section{Contributions} \label{sec:ae4}

TODO

\section{Dissertation Structure} \label{sec:struct}

In addition to the introduction, this dissertation contains 8 more chapters. Chapter 2 reviews the relevant background and related work in vulnerability detection and program repair. Chapter 3 presents the profiling of security faults and discusses patterns observed across real-world software. Chapter 4 describes the construction of the benchmark dataset used in subsequent experiments. Chapter 5 provides a comparative analysis of existing APR and SVD tools in terms of their vulnerability coverage and performance. Chapter 6 introduces and evaluates the hybrid repair approach combining APR with static vulnerability detection. Chapter 7 explores the integration of ML-based detection with SAST tools in a collaborative setting. Chapter 8 discusses the key findings, implications, and limitations of the work. Chapter 9 concludes the thesis and outlines directions for future research.